\documentclass[ngerman]{scrartcl}
\usepackage{microtype}
\usepackage[T1]{fontenc}
\usepackage[utf8]{inputenc}
\setlength{\parskip}{\medskipamount}
\setlength{\parindent}{0pt}
\usepackage{babel}
\usepackage{float}
\usepackage{amsmath}
\usepackage{amstext}
\usepackage{varioref}
\usepackage[unicode=true,pdfusetitle,
 bookmarks=true,bookmarksnumbered=false,bookmarksopen=false, breaklinks=false,pdfborder={0 0 0},backref=false,colorlinks=false] {hyperref}
\usepackage{graphicx}
\usepackage{setspace}
\usepackage{cleveref}
\usepackage{icomma}

% % % % % % % % % % % % % % % % % % % % %
%Schriften - nur eine auskommentieren
	%Latin Modern - Standard
		%\usepackage{lmodern}
	%Times Roman für strenge Dozenten
		%\usepackage{times}
	%Garamond - sieht toll aus, aber eher exotisch
		%\usepackage[cmintegrals,cmbraces]{newtxmath}\usepackage{ebgaramond-maths}\usepackage{helvet}

%Hurenkinder, Schusterjungen
\widowpenalties=3 10000 10000 150


% % % % % % % % % % % % % % % % % % % % %
%LITERATUR
%Fußnote _ODER_ Amerikanisch auskommentieren

%Amerikanische In-Text-Zitierweise
%\usepackage[backend=biber,style=authoryear,maxcitenames=1]{biblatex}\renewcommand{\cite}{\parencite}

%Deutsches Fußnotensystem
%\usepackage[backend=biber,style=authortitle-dw,firstfull,maxcitenames=1]{biblatex}\DeclareFieldFormat{title}{\mkbibemph{#1}}\DeclareFieldFormat{citetitle}{\mkbibemph{#1}}\renewcommand{\cite}{\footcite}

%Ressource
\addbibresource{/pfad/zur/.bib}
% % % % % % % % % % % % % % % % % % % % %
\makeatletter
\begin{document}
		%%Hier alle Daten einfügen
\newcommand{\autorname}{Name des Autors}
\newcommand{\autormail}{Email des Autors}
\newcommand{\autormatr}{Matrikelnummer des Autors}
\newcommand{\fak}{Fakultät}
\newcommand{\lehrstuhl}{Lehrstuhl}
\newcommand{\lv}{Lehrveranstaltung}
\newcommand{\semester}{Semester}
\newcommand{\lp}{Lehrperson}
\newcommand{\arbtyp}{Typ der Arbeit}
\newcommand{\titel}{Titel}

\title{\titel}
\author{\autorname}
		\begin{titlepage}
\vspace*{4cm}
\begin{tabu} to \textwidth {X[c]}
	\toprule
	\toprule[2pt]
	\huge{\textsc{\titel}}\\
	\Large{\textsc{\arbtyp}}\\
	\bottomrule[2pt]
	\bottomrule
\end{tabu}
\\[3cm]
\centering{
\textsc{\Large \autorname}
}
\\[3cm]
\begin{tabu} {>{\itshape}X[r] X[1.5]}
Fakultät & \fak\\
Lehrstuhl & \lehrstuhl\\
Lehrveranstaltung & \lv\\
Semester & \semester\\
Lehrperson & \lp \\
Fachmentor & Markus Semmler\\
\end{tabu}
\vfill
\includegraphics[width=2cm]{res/uni_potsdam_logo.pdf}

\end{titlepage}

\tableofcontents
\begin{center}\autorname $\cdot$ \autormatr $\cdot$ \href{mailto:\autormail}{\autormail}\end{center}
		\tableofcontents{}
		\pagebreak{}



\onehalfspacing
\input{/pfad/zum/inhalt}






		\printbibliography
		
		%\pagebreak
\thispagestyle{empty}
\section*{Eidesstattliche Erklärung}

Hiermit versichere ich an Eides statt, dass ich die vorliegende Arbeit selbstständig verfasst und keine anderen als die angegebenen Quellen und Hilfsmittel benutzt habe, alle Ausführungen, die anderen Schriften wörtlich oder sinngemäß entnommen wurden, kenntlich gemacht sind und die Arbeit in gleicher oder ähnlicher Fassung noch nicht Bestandteil einer Studien- oder Prüfungsleistung war.\\[5cm]

Potsdam, den \today \hfill \autorname
\end{document}
